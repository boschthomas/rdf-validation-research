% This is LLNCS.DEM the demonstration file of
% the LaTeX macro package from Springer-Verlag
% for Lecture Notes in Computer Science,
% version 2.4 for LaTeX2e as of 16. April 2010
%
\documentclass{llncs}

\usepackage[utf8]{inputenc}

% URL handling
\usepackage{url}
\urlstyle{same}

% Todos
%\usepackage[colorinlistoftodos]{todonotes}
%\newcommand{\ke}[1]{\todo[size=\small, color=orange!40]{\textbf{Kai:} #1}}
%\newcommand{\tb}[1]{\todo[size=\small, color=green!40]{\textbf{Thomas:} #1}}


%\usepackage{makeidx}  % allows for indexgeneration

%\usepackage{amsmath}
\usepackage{amsmath, amssymb}
\usepackage{mathabx}

% monospace within text
\newcommand{\ms}[1]{\texttt{#1}}

% examples
\usepackage{fancyvrb}
\DefineVerbatimEnvironment{ex}{Verbatim}{numbers=left,numbersep=2mm,frame=single,fontsize=\scriptsize}

\usepackage{xspace}
% Einfache und doppelte Anfuehrungszeichen
\newcommand{\qs}{``} 
\newcommand{\qe}{''\xspace} 
\newcommand{\sqs}{`} 
\newcommand{\sqe}{'\xspace} 

% checkmark
\usepackage{tikz}
\def\checkmark{\tikz\fill[scale=0.4](0,.35) -- (.25,0) -- (1,.7) -- (.25,.15) -- cycle;} 

% Xs
\usepackage{pifont}

% Tabellenabstände kleiner
\setlength{\intextsep}{10pt} % Vertical space above & below [h] floats
\setlength{\textfloatsep}{10pt} % Vertical space below (above) [t] ([b]) floats
% \setlength{\abovecaptionskip}{0pt}
% \setlength{\belowcaptionskip}{0pt}

\usepackage{tabularx}
\newcommand{\hr}{\hline\noalign{\smallskip}} % für die horizontalen linien in tabellen

% Todos
\usepackage[colorinlistoftodos]{todonotes}
\newcommand{\ke}[1]{\todo[size=\small, color=orange!40]{\textbf{Kai:} #1}}
\newcommand{\tb}[1]{\todo[size=\small, color=green!40]{\textbf{Thomas:} #1}}



\begin{document}

%
%
\title{XXXXX}
%
\titlerunning{XXXXX}  % abbreviated title (for running head)
%                                     also used for the TOC unless
%                                     \toctitle is used
%
\author{XXXXX\inst{1} \and XXXXX\inst{2}}
%
\authorrunning{XXXXX} % abbreviated author list (for running head)
%
%%%% list of authors for the TOC (use if author list has to be modified)
\institute{XXXXX\\
\email{XXXXX},\\ 
\and
XXXXX \\
\email{XXXXX} 
}

\maketitle              % typeset the title of the contribution

\begin{abstract}


\keywords{..}
\end{abstract}
%

\section{Introduction}



\section{Ideas}



\begin{itemize}
	\item sind alle constraints abgedeckt?
	\item kann man alle constraints in SPARQL definieren?
	\item sind alle constraints mit Logik ausdrückbar?
	\item vollständig mit Reasoning | OW
	\item vollständig ohne Reasoning | CW
	\item es gibt keinen query rewriting mechanismus für OWL 2, nur für OWL-QL
	\item constraints in einer anderen constraint language definieren wenn constraints nicht in OWL beschrieben werden können
	\item durch reasoning entstehen Probleme, auf die man nicht gekommen wäre --> sofort nachvollziehbar
	\item zeigen, dass OWL-QL und constraint language einer konkreten Domäne zusammen vollständig sind
	\item System entwickeln, das effizient ist / Experimente
\end{itemize}

-----

Nehmen wir nun an, dass dein Framework welches entsprechende SPARQL Queries generiert diese auf einem SPARQL Endpoint evaluiert der zu der vorliegenden Ontologie bzw. des darin verwendeten OWL 2 Profils das entsprechende Entailment Regime realisiert, wären die zurückgegebenen Resultsets vollständig. Wie das Entailment Regime im Endpoint realisiert ist, also durch Query Rewriting oder durch Vervollständigung der ABox, ist dabei irrelevant.

Wie allerdings bspw. in 
\url{https://www.uni-ulm.de/fileadmin/website_uni_ulm/iui.inst.090/Lehre/WS_2011-2012/SemWebGrundlagen/LectureNotes.pdf}
auf Seite 51 veranschaulicht, ist die Komplexität des Reasoning abhängig von der zugrunde gelegten Sprache und kann daher nur in bestimmten Fällen effizient durchgeführt werden. Wie in unserem letzten Paper beschrieben zielt unter anderem die Definition von DL-Lite gerade darauf ab Reasoning Aufgaben und Query Answering effizient zu ermöglichen und ist Grundlage des OWL 2 QL Profils. Nun ist allgemein bekannt, dass die logische Konsistenz für diese Art von Sprachen effizient geprüft werden kann. 

Allerding wäre wie bspw. in 
\url{http://www.aifb.kit.edu/images/d/d2/2005_925_Haase_Consistent_Evol_1.pdf} beschrieben auch eine sogenannte 'User-defined Consistency' denkbar. Genau an dieser Stelle könnten wir ansetzen.

\section{research questions}

\begin{itemize}
	\item for which RDF validation requirement the expressivity of DL-LiteA respectively OWL 2 QL is sufficient?
	\item for which RDF validation requirement additional constraint languages are needed?
	\item which constraint languages are suitable to express remaining requirements?
	\item what are the effects of these constraints regarding complexity?
\end{itemize}

\section{OWL 2 QL}

OWL 2 profiles specification: \cite{owl2profiles2008}

\begin{itemize}
  \item OWL 2 QL constructs
	\item Difference between OWL 2 DL and OWL 2 QL
\end{itemize}

\textbf{Logical Underpinning for OWL 2 QL.}
OWL 2 QL is based on the DL-Lite family of description logics. Several variants of DL-Lite have been described in the literature, and DL-Lite$_R$ provides the logical underpinning for OWL 2 QL. DL-Lite$_R$ does not require the unique name assumption (UNA), since making this assumption would have no impact on the semantic consequences of a DL-Lite$_R$ ontology. More expressive variants of DL-Lite, such as DL-Lite$_A$, extend DL-Lite$_R$ with functional properties, and these can also be extended with keys; however, for query answering to remain in LOGSPACE, these extensions require UNA and need to impose certain global restrictions on the interaction between properties used in different types of axiom. Basing OWL 2 QL on DL-Lite$_R$ avoids practical problems involved in the explicit axiomatization of UNA \cite{owl2profiles2008}. 

\section{RDF Validation Requirements Not Covered By OWL 2 QL}

\subsection{Class-Specific Disjointness of Properties}

requirements:

\begin{itemize}
	\item R-11-DISJOINT-DATA-PROPERTIES-CLASS-SPECIFIC
  \item R-12-DISJOINT-OBJECT-PROPERTIES-CLASS-SPECIFIC
  \item R-13-DISJOINT-GROUP-OF-PROPERTIES-CLASS-SPECIFIC
\end{itemize}

\textbf{class-specific disjoint data properties}

constraint (ShEx):

\begin{ex}
(foaf:name xsd:string  | foaf:givenName xsd:string+, foaf:familyName xsd:string)
\end{ex}

\textbf{class-specific disjoint object properties}



\textbf{Class-Specific Disjoint Group of Properties}

requirements DB:
\begin{itemize}
  \item ID: R-13-DISJOINT-GROUP-OF-PROPERTIES-CLASS-SPECIFIC
	\item URL: \url{http://lelystad.informatik.uni-mannheim.de/rdf-validation/?q=node/20}
\end{itemize}

A \textless Human\textgreater has either a name or at least 1 given name and 1 family name.

constraint (ShEx):

\begin{ex}
<Human> {                           
    (                                    
            foaf:name xsd:string              
        |                                   
            foaf:givenName xsd:string+,      
            foaf:familyName xsd:string
    )
}
\end{ex}

valid data (ShEx):

\begin{ex}
<Anakin>
    foaf:givenName "Anakin" ;
    foaf:familyName "Skywalker" .
\end{ex}

\begin{ex}
<Anakin>
    foaf:name "Anakin Skywalker" .
\end{ex}

invalid data (ShEx):

\begin{ex}
<Anakin>
    foaf:givenName "Anakin" ;
    foaf:familyName "Skywalker" ;
    foaf:name "Anakin Skywalker" .
\end{ex}

\subsection{Default Values}

requirements:

\begin{itemize}
	\item R-31-DEFAULT-VALUES-OF-RDF-OBJECTS
  \item R-38-DEFAULT-VALUES-OF-RDF-LITERALS
\end{itemize}

\subsection{Membership in Controlled Vocabularies}

requirements:

\begin{itemize}
	\item R-32-MEMBERSHIP-OF-RDF-OBJECTS-IN-CONTROLLED-VOCABULARIES
	\item R-39-MEMBERSHIP-OF-RDF-LITERALS-IN-CONTROLLED-VOCABULARIES
\end{itemize}

\subsection{Pattern Matching}

requirements:

\begin{itemize}
	\item R-21-IRI-PATTERN-MATCHING-ON-RDF-SUBJECTS
  \item R-22-IRI-PATTERN-MATCHING-ON-RDF-OBJECTS
  \item R-23-IRI-PATTERN-MATCHING-ON-RDF-PROPERTIES
  \item R-44-PATTERN-MATCHING-ON-RDF-LITERALS
  \item R-141-NEGATIVE-PATTERN-MATCHING-ON-RDF-LITERALS
\end{itemize}

\subsection{Calculations on and Comparisons of RDF Literals}

requirements:

\begin{itemize}
	\item R-41-STATISTICAL-COMPUTATIONS
	\item R-42-COMPUTATIONS-BASED-ON-DATATYPE
	\item R-43-COMPARISONS-BASED-ON-DATATYPE
\end{itemize}

\subsection{Constraining Facets on RDF Literals}

requirements:

\begin{itemize}
	\item R-45-RANGES-OF-RDF-LITERAL-VALUES
  \item R-142-NEGATIVE-RANGES-OF-RDF-LITERAL-VALUES
	\item R-46-CONSTRAINING-FACETS
	\item R-50-WHITESPACE-HANDLING-OF-RDF-LITERALS
\end{itemize}

\subsection{Language of RDF Literals}

requirements:

\begin{itemize}
	\item R-47-LANGUAGE-TAG-MATCHING
  \item R-48-MISSING-LANGUAGE-TAGS
	\item R-49-RDF-LITERALS-HAVING-AT-MOST-ONE-LANGUAGE-TAG
\end{itemize}

\subsection{Property Occurrences}

requirements:

\begin{itemize}
	\item R-52-NEGATIVE-OBJECT-PROPERTY-CONSTRAINTS
	\item R-53-NEGATIVE-DATA-PROPERTY-CONSTRAINTS
	\item R-67-CLASSIFY-PROPERTIES-ACCORDING-TO-OCCURRENCE
\end{itemize}

\subsection{Property Groups}

requirements:

\begin{itemize}
	\item R-13-DISJOINT-GROUP-OF-PROPERTIES-CLASS-SPECIFIC
	\item R-66-PROPERTY-GROUPS
\end{itemize}



\section{Related Work}



\section{Evaluation}

evaluation using practical data set

\section{Conclusion and Future Work}

\section{Appendix}


\subsection{Allowed Usage of Constructs in Class Expressions in OWL 2 QL}

\textbf{Subclass Expressions}

\begin{ex}
subClassExpression :=
    Class |
    subObjectSomeValuesFrom | DataSomeValuesFrom
subObjectSomeValuesFrom := 'ObjectSomeValuesFrom' '(' ObjectPropertyExpression owl:Thing ')'
\end{ex}

\textbf{Superclass Expressions}

\begin{ex}
superClassExpression :=
    Class |
    superObjectIntersectionOf | superObjectComplementOf |
    superObjectSomeValuesFrom | DataSomeValuesFrom
\end{ex}
 
\subsection{Supported Constructs in OWL 2 QL}

\begin{itemize}
	\item subclass axioms (SubClassOf)
  \item class expression equivalence (EquivalentClasses)
  \item class expression disjointness (DisjointClasses)
  \item inverse object properties (InverseObjectProperties)
  \item property inclusion (SubObjectPropertyOf not involving property chains and SubDataPropertyOf)
  \item property equivalence (EquivalentObjectProperties and EquivalentDataProperties)
  \item property domain (ObjectPropertyDomain and DataPropertyDomain)
  \item property range (ObjectPropertyRange and DataPropertyRange)
  \item disjoint properties (DisjointObjectProperties and DisjointDataProperties)
  \item symmetric properties (SymmetricObjectProperty)
  \item reflexive properties (ReflexiveObjectProperty)
  \item irreflexive properties (IrreflexiveObjectProperty)
  \item asymmetric properties (AsymmetricObjectProperty)
  \item assertions other than individual equality assertions and negative property assertions (DifferentIndividuals, ClassAssertion, ObjectPropertyAssertion, and DataPropertyAssertion) 
\end{itemize}

\subsection{Not Supported Constructs in OWL 2 QL}

\begin{itemize}
	\item existential quantification to a class expression or a data range (ObjectSomeValuesFrom and DataSomeValuesFrom) in the subclass position
  \item self-restriction (ObjectHasSelf)
  \item existential quantification to an individual or a literal (ObjectHasValue, DataHasValue)
  \item enumeration of individuals and literals (ObjectOneOf, DataOneOf)
  \item universal quantification to a class expression or a data range (ObjectAllValuesFrom, DataAllValuesFrom)
  \item cardinality restrictions (ObjectMaxCardinality, ObjectMinCardinality, ObjectExactCardinality, DataMaxCardinality, DataMinCardinality, DataExactCardinality)
  \item disjunction (ObjectUnionOf, DisjointUnion, and DataUnionOf)
  \item property inclusions (SubObjectPropertyOf) involving property chains
  \item functional and inverse-functional properties (FunctionalObjectProperty, InverseFunctionalObjectProperty, and FunctionalDataProperty)
  \item transitive properties (TransitiveObjectProperty)
  \item keys (HasKey)
  \item individual equality assertions and negative property assertions
\end{itemize}

\subsection{RDF Validation Requirements Covered by OWL 2 QL}

\begin{table}
\caption{RDF Validation Requirements Covered by OWL 2 QL}
\label{tab:RequirementsCoveredOWL2QL}
\centering
\begin{tabular}{ll}
\hr
Requirements Classification & Requirements \\
\hr
& R-1-UNIQUENESS-OF-URIS \\
& R-2-UNIQUE-INSTANCES \\
& R-3-EQUIVALENT-CLASSES (EquivalentClasses) \\
& R-4-EQUIVALENT-OBJECT-PROPERTIES (EquivalentObjectProperties) \\
& R-5-EQUIVALENT-DATA-PROPERTIES (EquivalentDataProperties) \\
& R-7-DISJOINT-CLASSES (DisjointClasses) \\
& R-9-DISJOINT-OBJECT-PROPERTIES (DisjointObjectProperties) \\
& R-10-DISJOINT-DATA-PROPERTIES (DisjointDataProperties) \\
& R-14-DISJOINT-INDIVIDUALS (DifferentIndividuals) \\
& R-25-OBJECT-PROPERTY-DOMAIN (ObjectPropertyDomain) \\
& R-26-DATA-PROPERTY-DOMAIN (DataPropertyDomain) \\
& R-27-CLASS-SPECIFIC-VALIDATION \\
& R-28-OBJECT-PROPERTY-RANGE (ObjectPropertyRange) \\
& R-35-DATA-PROPERTY-RANGE (DataPropertyRange) \\
& R-54-SUB-OBJECT-PROPERTIES (SubObjectPropertyOf) \\
& R-56-INVERSE-OBJECT-PROPERTIES (ObjectInverseOf) \\
& R-59-REFLEXIVE-OBJECT-PROPERTIES (ReflexiveObjectProperty) \\
& R-60-IRREFLEXIVE-OBJECT-PROPERTIES (IrreflexiveObjectProperty) \\
& R-61-SYMMETRIC-OBJECT-PROPERTIES (SymmetricObjectProperty) \\
& R-62-ASYMMETRIC-OBJECT-PROPERTIES (AsymmetricObjectProperty) \\
& R-64-SUB-DATA-PROPERTIES (SubDataPropertyOf) \\
\hr
\end{tabular}
\end{table}

\subsection{RDF Validation Requirements Not Covered by OWL 2 QL}

\begin{table}
\caption{RDF Validation Requirements Not Covered by OWL 2 QL}
\label{tab:RequirementsNotCoveredOWL2QL}
\centering
\begin{tabular}{lc}
\hr
Requirements & Covering Constraint Languages \\
\hr
R-6-EQUIVALENT-INDIVIDUALS (SameIndividual) & OWL 2 DL \\
R-8-DISJOINT-UNION-OF-CLASS-EXPRESSIONS (DisjointUnion) & OWL 2 DL \\
R-11-DISJOINT-DATA-PROPERTIES-CLASS-SPECIFIC & ShEx \\
R-12-DISJOINT-OBJECT-PROPERTIES-CLASS-SPECIFIC & ShEx \\
R-13-DISJOINT-GROUP-OF-PROPERTIES-CLASS-SPECIFIC & ShEx \\
R-15-CONJUNCTION-OF-CLASS-EXPRESSIONS (ObjectIntersectionOf) & OWL 2 DL \\
R-16-CONJUNCTION-OF-DATA-RANGES (DataIntersectionOf) & OWL 2 DL \\
R-17-DISJUNCTION-OF-CLASS-EXPRESSIONS (ObjectUnionOf) & OWL 2 DL \\
R-18-DISJUNCTION-OF-DATA-RANGES (DataUnionOf) & OWL 2 DL \\
R-19-NEGATION-OF-CLASS-EXPRESSIONS (ObjectComplementOf) & OWL 2 DL \\
R-20-NEGATION-OF-DATA-RANGES (DataComplementOf) & OWL 2 DL \\
R-21-IRI-PATTERN-MATCHING-ON-RDF-SUBJECTS & SPIN \\
R-22-IRI-PATTERN-MATCHING-ON-RDF-OBJECTS & SPIN, ShEx \\
R-23-IRI-PATTERN-MATCHING-ON-RDF-PROPERTIES & SPIN, ShEx \\
R-24-PROVENANCE-CONSTRAINTS & \\
R-29-CLASS-SPECIFIC-RANGE-OF-RDF-OBJECTS & SPIN, RS \\
R-30-ALLOWED-VALUES-FOR-RDF-OBJECTS (ObjectOneOf) & OWL 2 DL \\
R-31-DEFAULT-VALUES-OF-RDF-OBJECTS & SPIN, BF, RS \\
R-32-MEMBERSHIP-OF-RDF-OBJECTS-IN-CONTROLLED-VOCABULARIES & DSP, SPIN \\
R-33-NEGATIVE-OBJECT-CONSTRAINTS \\
R-34-AVAILABLE-CLASS-DEFINITION & ShEx, SPIN \\
R-36-CLASS-SPECIFIC-RANGE-OF-RDF-LITERALS & ShEx, BF, SPIN \\
R-37-ALLOWED-VALUES-FOR-RDF-LITERALS (DataOneOf) & OWL 2 DL \\
R-38-DEFAULT-VALUES-OF-RDF-LITERALS & SPIN, BF, RS \\
R-39-MEMBERSHIP-OF-RDF-LITERALS-IN-CONTROLLED-VOCABULARIES & DSP, SPIN \\
R-44-PATTERN-MATCHING-ON-RDF-LITERALS & ShEx, RS, SPIN \\
R-41-STATISTICAL-COMPUTATIONS & SPIN \\
R-42-COMPUTATIONS-BASED-ON-DATATYPE & \\
R-43-COMPARISONS-BASED-ON-DATATYPE & DQTP, ShEx, SPIN \\
R-45-RANGES-OF-RDF-LITERAL-VALUES & DQTP, SPIN \\
R-46-CONSTRAINING-FACETS & Stardog, SPIN \\
R-47-LANGUAGE-TAG-MATCHING & SPIN \\
R-48-MISSING-LANGUAGE-TAGS & SPIN \\
R-49-RDF-LITERALS-HAVING-AT-MOST-ONE-LANGUAGE-TAG & DQTP, SPIN \\
R-50-WHITESPACE-HANDLING-OF-RDF-LITERALS & SPIN \\
R-51-HTML-HANDLING-OF-RDF-LITERALS & SPIN \\
R-52-NEGATIVE-OBJECT-PROPERTY-CONSTRAINTS & ShEx, SPIN \\
R-53-NEGATIVE-DATA-PROPERTY-CONSTRAINTS & ShEx, SPIN \\
R-55-OBJECT-PROPERTY-PATHS (SubObjectPropertyOf) & OWL 2 DL \\
R-57-FUNCTIONAL-OBJECT-PROPERTIES (FunctionalObjectProperty) & OWL 2 DL \\
R-58-INVERSE-FUNCTIONAL-OBJECT-PROPERTIES (InverseFunctionalObjectProperty) & OWL 2 DL \\
R-63-TRANSITIVE-OBJECT-PROPERTIES (TransitiveObjectProperty) & OWL 2 DL \\
R-65-FUNCTIONAL-DATA-PROPERTIES (FunctionalDataProperty) & OWL 2 DL \\
R-66-PROPERTY-GROUPS & ShEx, SPIN \\
R-67-CLASSIFY-PROPERTIES-ACCORDING-TO-OCCURRENCE & \\

R-141-NEGATIVE-PATTERN-MATCHING-ON-RDF-LITERALS & DQTP, SPIN \\
R-142-NEGATIVE-RANGES-OF-RDF-LITERAL-VALUES & DQTP, SPIN \\
\hr
\end{tabular}
\end{table}

\bibliography{literatur}{}
\bibliographystyle{plain}
\setcounter{tocdepth}{1}
%\listoftodos
\end{document}
