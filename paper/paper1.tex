% This is LLNCS.DEM the demonstration file of
% the LaTeX macro package from Springer-Verlag
% for Lecture Notes in Computer Science,
% version 2.4 for LaTeX2e as of 16. April 2010
%
\documentclass{llncs}

% allows for temporary adjustment of side margins
\usepackage{chngpage}

% just makes the table prettier (see \toprule, \bottomrule, etc. commands below)
\usepackage{booktabs}

\usepackage[utf8]{inputenc}

% URL handling
\usepackage{url}
\urlstyle{same}

% Todos
%\usepackage[colorinlistoftodos]{todonotes}
%\newcommand{\ke}[1]{\todo[size=\small, color=orange!40]{\textbf{Kai:} #1}}
%\newcommand{\tb}[1]{\todo[size=\small, color=green!40]{\textbf{Thomas:} #1}}


%\usepackage{makeidx}  % allows for indexgeneration

%\usepackage{amsmath}
\usepackage{amsmath, amssymb}
\usepackage{mathabx}

% monospace within text
\newcommand{\ms}[1]{\texttt{#1}}

% examples
\usepackage{fancyvrb}
\DefineVerbatimEnvironment{ex}{Verbatim}{numbers=left,numbersep=2mm,frame=single,fontsize=\scriptsize}

\usepackage{xspace}
% Einfache und doppelte Anfuehrungszeichen
\newcommand{\qs}{``} 
\newcommand{\qe}{''\xspace} 
\newcommand{\sqs}{`} 
\newcommand{\sqe}{'\xspace} 

% checkmark
\usepackage{tikz}
\def\checkmark{\tikz\fill[scale=0.4](0,.35) -- (.25,0) -- (1,.7) -- (.25,.15) -- cycle;} 

% Xs
\usepackage{pifont}

% Tabellenabstände kleiner
\setlength{\intextsep}{10pt} % Vertical space above & below [h] floats
\setlength{\textfloatsep}{10pt} % Vertical space below (above) [t] ([b]) floats
% \setlength{\abovecaptionskip}{0pt}
% \setlength{\belowcaptionskip}{0pt}

\usepackage{tabularx}
\newcommand{\hr}{\hline\noalign{\smallskip}} % für die horizontalen linien in tabellen

% Todos
\usepackage[colorinlistoftodos]{todonotes}
\newcommand{\ke}[1]{\todo[size=\small, color=orange!40]{\textbf{Kai:} #1}}
\newcommand{\tb}[1]{\todo[size=\small, color=green!40]{\textbf{Thomas:} #1}}

\setcounter{secnumdepth}{5}

\begin{document}

%
%
\title{XXXXX}
%
\titlerunning{XXXXX}  % abbreviated title (for running head)
%                                     also used for the TOC unless
%                                     \toctitle is used
%
\author{XXXXX\inst{1} \and XXXXX\inst{2}}
%
\authorrunning{XXXXX} % abbreviated author list (for running head)
%
%%%% list of authors for the TOC (use if author list has to be modified)
\institute{XXXXX\\
\email{XXXXX},\\ 
\and
XXXXX \\
\email{XXXXX} 
}

\maketitle              % typeset the title of the contribution

\begin{abstract}


\keywords{..}
\end{abstract}
%

\section{Introduction}



\section{Ideas}



\begin{itemize}
	\item sind alle constraints abgedeckt?
	\item kann man alle constraints in SPARQL definieren?
	\item sind alle constraints mit Logik ausdrückbar?
	\item vollständig mit Reasoning | OW
	\item vollständig ohne Reasoning | CW
	\item es gibt keinen query rewriting mechanismus für OWL 2, nur für OWL-QL
	\item constraints in einer anderen constraint language definieren wenn constraints nicht in OWL beschrieben werden können
	\item durch reasoning entstehen Probleme, auf die man nicht gekommen wäre --> sofort nachvollziehbar
	\item zeigen, dass OWL-QL und constraint language einer konkreten Domäne zusammen vollständig sind
	\item System entwickeln, das effizient ist / Experimente
\end{itemize}

-----

Nehmen wir nun an, dass dein Framework welches entsprechende SPARQL Queries generiert diese auf einem SPARQL Endpoint evaluiert der zu der vorliegenden Ontologie bzw. des darin verwendeten OWL 2 Profils das entsprechende Entailment Regime realisiert, wären die zurückgegebenen Resultsets vollständig. Wie das Entailment Regime im Endpoint realisiert ist, also durch Query Rewriting oder durch Vervollständigung der ABox, ist dabei irrelevant.

Wie allerdings bspw. in 
\url{https://www.uni-ulm.de/fileadmin/website_uni_ulm/iui.inst.090/Lehre/WS_2011-2012/SemWebGrundlagen/LectureNotes.pdf}
auf Seite 51 veranschaulicht, ist die Komplexität des Reasoning abhängig von der zugrunde gelegten Sprache und kann daher nur in bestimmten Fällen effizient durchgeführt werden. Wie in unserem letzten Paper beschrieben zielt unter anderem die Definition von DL-Lite gerade darauf ab Reasoning Aufgaben und Query Answering effizient zu ermöglichen und ist Grundlage des OWL 2 QL Profils. Nun ist allgemein bekannt, dass die logische Konsistenz für diese Art von Sprachen effizient geprüft werden kann. 

Allerding wäre wie bspw. in 
\url{http://www.aifb.kit.edu/images/d/d2/2005_925_Haase_Consistent_Evol_1.pdf} beschrieben auch eine sogenannte 'User-defined Consistency' denkbar. Genau an dieser Stelle könnten wir ansetzen.

\section{research questions}

\begin{itemize}
	\item for which RDF validation requirement the expressivity of DL-LiteA respectively OWL 2 QL is sufficient?
	\item for which RDF validation requirement additional constraint languages are needed?
	\item which constraint languages are suitable to express remaining requirements?
	\item what are the effects of these constraints regarding complexity?
\end{itemize}

\section{OWL 2 QL}

OWL 2 profiles specification: \cite{owl2profiles2008}

\begin{itemize}
  \item OWL 2 QL constructs
	\item Difference between OWL 2 DL and OWL 2 QL
\end{itemize}

\textbf{Logical Underpinning for OWL 2 QL.}
OWL 2 QL is based on the DL-Lite family of description logics. Several variants of DL-Lite have been described in the literature, and DL-Lite$_R$ provides the logical underpinning for OWL 2 QL. DL-Lite$_R$ does not require the unique name assumption (UNA), since making this assumption would have no impact on the semantic consequences of a DL-Lite$_R$ ontology. More expressive variants of DL-Lite, such as DL-Lite$_A$, extend DL-Lite$_R$ with functional properties, and these can also be extended with keys; however, for query answering to remain in LOGSPACE, these extensions require UNA and need to impose certain global restrictions on the interaction between properties used in different types of axiom. Basing OWL 2 QL on DL-Lite$_R$ avoids practical problems involved in the explicit axiomatization of UNA \cite{owl2profiles2008}. 

\section{RDF Validation Requirements Not Covered By OWL 2 QL}

\subsection{Class-Specific Disjointness of Properties}

requirements:

\begin{itemize}
	\item R-11-DISJOINT-DATA-PROPERTIES-CLASS-SPECIFIC
  \item R-12-DISJOINT-OBJECT-PROPERTIES-CLASS-SPECIFIC
  \item R-13-DISJOINT-GROUP-OF-PROPERTIES-CLASS-SPECIFIC
\end{itemize}

\subsubsection{Class-Specific Disjoint Data Properties}

R-11-DISJOINT-DATA-PROPERTIES-CLASS-SPECIFIC

\begin{itemize}
	\item exclusive OR of data properties
	\item with OWL 2, inclusive OR of properties would be possible to express, but not exclusive OR of data properties
\end{itemize}

constraint (ShEx):

\begin{ex}
(foaf:name xsd:string  | foaf:givenName xsd:string+, foaf:familyName xsd:string)
\end{ex}

valid data (ShEx):

\begin{ex}
<Anakin>
    foaf:name "Anakin Skywalker" ;
		foaf:familyName "Skywalker" .
\end{ex}

\begin{ex}
<Anakin>
    foaf:givenName "Anakin" ;
		foaf:givenName "Darth" ;
    foaf:familyName "Skywalker" .
\end{ex}

invalid data (ShEx):

\begin{ex}
<Anakin>
    foaf:name "Anakin Skywalker" ;
    foaf:givenName "Anakin" ;
    foaf:familyName "Skywalker" .
\end{ex}

\subsubsection{Class-Specific Disjoint Object Properties}

R-12-DISJOINT-OBJECT-PROPERTIES-CLASS-SPECIFIC

\begin{itemize}
	\item see R-11-DISJOINT-DATA-PROPERTIES-CLASS-SPECIFIC
\end{itemize}

\subsubsection{Class-Specific Disjoint Group of Properties}

R-13-DISJOINT-GROUP-OF-PROPERTIES-CLASS-SPECIFIC

\begin{itemize}
	\item exclusive OR of property groups
	\item with OWL 2, inclusive OR of property groups would be possible to express, but not exclusive OR of property groups
\end{itemize}

A \textless Human\textgreater has either a name or at least 1 given name and 1 family name.

constraint (ShEx):

\begin{ex}
<Human> {                           
    (                                    
            foaf:name xsd:string              
        |                                   
            foaf:givenName xsd:string+,      
            foaf:familyName xsd:string
    )
}
\end{ex}

valid data (ShEx):

\begin{ex}
<Anakin>
    foaf:givenName "Anakin" ;
    foaf:familyName "Skywalker" .
\end{ex}

\begin{ex}
<Anakin>
    foaf:name "Anakin Skywalker" .
\end{ex}

invalid data (ShEx):

\begin{ex}
<Anakin>
    foaf:givenName "Anakin" ;
    foaf:familyName "Skywalker" ;
    foaf:name "Anakin Skywalker" .
\end{ex}

\subsection{Default Values}

requirements:

\begin{itemize}
	\item R-31-DEFAULT-VALUES-OF-RDF-OBJECTS
  \item R-38-DEFAULT-VALUES-OF-RDF-LITERALS
\end{itemize}

\subsubsection{R-31-DEFAULT-VALUES-OF-RDF-OBJECTS}

\begin{itemize}
	\item see R-38-DEFAULT-VALUES-OF-RDF-LITERALS
\end{itemize}

\subsubsection{R-38-DEFAULT-VALUES-OF-RDF-LITERALS}

rule (SPIN):

Jedis have only 1 blue laser sword per default.
Siths, in contrast, normally have 2 red laser swords.

\begin{ex}
owl:Thing
    spin:rule [
        a sp:Construct ;
            sp:text """
                CONSTRUCT {            
                    ?this :laserSwordColor "blue"^^xsd:string ;
                    ?this :numberLaserSwords "1"^^xsd:nonNegativeInteger . 
                }
                WHERE {             
                    ?this a :Jedi .            
                } """ ; ] .
owl:Thing
    spin:rule [
        a sp:Construct ;
            sp:text """
                CONSTRUCT {
                    ?this :laserSwordColor "red"^^xsd:string ;
                    ?this :numberLaserSwords "2"^^xsd:nonNegativeInteger . 
                }
                WHERE {             
                    ?this a :Sith .            
                } """ ; ] .
\end{ex}

data (SPIN):

\begin{ex}
:Joda a :Jedi .
:DarthSidious a :Sith .
\end{ex}

inferred triples (SPIN):

\begin{ex}
:Joda 
    :laserSwordColor "blue"^^xsd:string ;
    :numberLaserSwords "1"^^xsd:nonNegativeInteger .
:DarthSidious 
    :laserSwordColor "red"^^xsd:string ;
    :numberLaserSwords "2"^^xsd:nonNegativeInteger .
\end{ex}

\subsection{Membership in Controlled Vocabularies}

requirements:

\begin{itemize}
	\item R-32-MEMBERSHIP-OF-RDF-OBJECTS-IN-CONTROLLED-VOCABULARIES
	\item R-39-MEMBERSHIP-OF-RDF-LITERALS-IN-CONTROLLED-VOCABULARIES
\end{itemize}

constraint (DSP):

\begin{ex}
:bookDescriptionTemplate 
    a dsp:DescriptionTemplate ;
    dsp:resourceClass swrc:Book ; 
    dsp:statementTemplate [
        a dsp:NonLiteralStatementTemplate ;
        dsp:property dcterms:subject ; 
        dsp:nonLiteralConstraint [ 
            a dsp:NonLiteralConstraint ;
            dsp:valueClass skos:Concept ; 
            dsp:vocabularyEncodingScheme :BookSubjects, :BookTopics, :BookCategories ] ] .
\end{ex}

A DSP consists of \ms{dsp:DescriptionTemplate}s that put constraints on instances of a certain class (\ms{dsp:resourceClass}). 
\ms{:bookDescriptionTemplate} describes resources of the type \ms{swrc:Book}.
The constraints can either be constraints on the description itself, e.g. a minimum occurrence of instances of this class. Additionally, constraints on single properties can be defined within a \ms{dsp:StatementTemplate}.
The \ms{dsp:NonLiteralStatementTemplate} restricts books to have \ms{dcterms:subject} (\ms{dsp:property}) relationships to RDF objects which are further described by the \ms{dsp:NonLiteralConstraint}.
These RDF objects have to be of the class \ms{skos:Concept} (\ms{dsp:ValueClass}).
Controlled vocabularies (like \ms{:BookSubjects}) are represented as \ms{skos:ConceptSchemes} in RDF and as \ms{dsp:VocabularyEncodingScheme}s in DSP.
\ms{dsp:VocabularyEncodingScheme} points to a list of controlled vocabularies the \ms{skos:Concept} resources must be members of.
The controlled vocabulary members (\ms{skos:Concept}s) are related to the controlled vocabulary (\ms{skos:ConceptScheme}) via the object properties \ms{skos:inScheme} and \ms{dcam:memberOf}.

valid data (DSP):

\begin{ex}
:ArtficialIntelligence
    a swrc:Book ;
    dcterms:subject :ComputerScience .
:ComputerScience
    a skos:Concept ;
    dcam:memberOf :BookSubjects ;
    skos:inScheme :BookSubjects .
:BookSubjects
    a skos:ConceptScheme .
\end{ex}

invalid data (DSP):

\begin{ex}
:ArtficialIntelligence
    a swrc:Book ;
    dcterms:subject :ComputerScience .
:ComputerScience
    a skos:Concept ;
    dcam:memberOf :BooksAboutBirds ;
    skos:inScheme :BooksAboutBirds ;
    dcam:memberOf :BookSubjects ;
    skos:inScheme :BookSubjects .
:BookSubjects
    a skos:ConceptScheme .
\end{ex}

The related subject (\ms{:ComputerScience}) is a member of a controlled vocabulary (\ms{:BooksAboutBirds}) 
which is not part of the list of allowed controlled vocabularies.

\subsection{Pattern Matching}

requirements:

\begin{itemize}
	\item R-21-IRI-PATTERN-MATCHING-ON-RDF-SUBJECTS
  \item R-22-IRI-PATTERN-MATCHING-ON-RDF-OBJECTS
  \item R-23-IRI-PATTERN-MATCHING-ON-RDF-PROPERTIES
  \item R-44-PATTERN-MATCHING-ON-RDF-LITERALS
  \item R-141-NEGATIVE-PATTERN-MATCHING-ON-RDF-LITERALS
\end{itemize}

\subsubsection{R-44-PATTERN-MATCHING-ON-RDF-LITERALS}

Covering approaches:

DQTP (MATCH Pattern), OWL 2 DL, RS, ShEx, SPARQL, SPIN

constraints (OWL 2 DL) [description logics abstract syntax]:

\begin{ex}

\end{ex}

constraints (OWL 2 DL) [functional-style syntax]:

\begin{ex}
Declaration( Datatype( :SSN ) ) 
DatatypeDefinition( 
    :SSN
    DatatypeRestriction( xsd:string xsd:pattern "[0-9]{3}-[0-9]{2}-[0-9]{4}" ) )     
DataPropertyRange( :hasSSN :SSN ) 
\end{ex}

constraints (OWL 2 DL) [turtle syntax]:

\begin{ex}
:SSN 
    a rdfs:Datatype ;
    owl:equivalentClass [
        a rdfs:Datatype ;
        owl:onDatatype xsd:string ;
        owl:withRestrictions ( 
            [ xsd:pattern "[0-9]{3}-[0-9]{2}-[0-9]{4}" ] ) ] .
\end{ex}

\begin{itemize}
	\item OWL 2 construct 'DatatypeRestriction' not allowed for OWL 2 QL
\end{itemize}

A social security number is a string that matches the given regular expression. 
The second axiom defines :SSN as an abbreviation for a datatype restriction on xsd:string. 
The first axiom explicitly declares :SSN to be a datatype. 
The datatype :SSN can be used just like any other datatype; 
for example, it is used in the third axiom to define the range of the :hasSSN property. 

valid data (OWL 2 DL):

\begin{ex}
:IimBernersLee
    :hasSSN "123-45-6789"^^:SSN .
\end{ex}

invalid data (OWL 2 DL):

\begin{ex}
:IimBernersLee
    :hasSSN "123456789"^^:SSN .
\end{ex}

\subsubsection{R-141-NEGATIVE-PATTERN-MATCHING-ON-RDF-LITERALS}

Covering approaches: DQTP (MATCH Pattern), OWL 2 DL, SPARQL, SPIN

constraints (DQTP):

MATCH Pattern \cite{Kontokostas2014} 

Application logic or real world constraints may put restrictions on the form of a literal value.
P1 is the property we need to check against REGEX and
NOP can be a not operator (!) or empty.

\begin{ex}
SELECT DISTINCT ?s WHERE { ?s %%P1%% ?value .
    FILTER ( %%NOP%% regex(str(?value), %%REGEX%) ) }
\end{ex}

example test bindings (DQTP):

\begin{ex}
1. dbo:isbn format is different ’!’ from '^([iIsSbBnN 0-9-])*$'
2. dbo:postCode format is different ‘!’ from 'ˆ[0-9]{5}$'
3. foaf:phone contains any letters ('[A-Za-z]')
\end{ex}

test binding (DQTP):

\begin{ex}
dbo:isbn format is different ’!’ from '^([iIsSbBnN 0-9-])*$'

P1 => dbo:isbn
NOP => !
REGEX => 'ˆ([iIsSbBnN 0-9-])*$'
\end{ex}

valid data (DQTP):

\begin{ex}
:FoundationsOfSWTechnologies
    dbo:isbn 'ISBN-13 978-1420090505' .
\end{ex}

invalid data (DQTP):

\begin{ex}
:HandbookOfSWTechnologies
    dbo:isbn 'DOI 10.1007/978-3-540-92913-0' .
\end{ex}

\subsection{Calculations on and Comparisons of RDF Literals}

requirements:

\begin{itemize}
	\item R-41-STATISTICAL-COMPUTATIONS
	\item R-42-COMPUTATIONS-BASED-ON-DATATYPE
	\item R-43-COMPARISONS-BASED-ON-DATATYPE
	\item R-194-PROVIDE-STRING-FUNCTIONS-FOR-RDF-LITERALS
\end{itemize}

\subsection{Constraining Facets on RDF Literals}

XSD constraining facets: \url{http://www.w3.org/TR/2001/REC-xmlschema-2-20010502/#rf-facets}

requirements:

\begin{itemize}
	\item R-44-PATTERN-MATCHING-ON-RDF-LITERALS
	\item R-45-RANGES-OF-RDF-LITERAL-VALUES
	\item R-46-CONSTRAINING-FACETS
	\item R-50-WHITESPACE-HANDLING-OF-RDF-LITERALS
	\item R-142-NEGATIVE-RANGES-OF-RDF-LITERAL-VALUES
\end{itemize}

\subsubsection{R-44-PATTERN-MATCHING-ON-RDF-LITERALS}

\begin{itemize}
	\item see 'Pattern Matching'
\end{itemize}

\subsubsection{R-45-RANGES-OF-RDF-LITERAL-VALUES}

constraints (OWL 2 DL) [description logics abstract syntax]:

\begin{ex}
ToDO
\end{ex}

constraints (OWL 2 DL) [functional-style syntax]:

\begin{ex}
Declaration( Datatype( :NumberPlayersPerWorldCupTeam ) ) 
DatatypeDefinition( 
    :NumberPlayersPerWorldCupTeam
    DatatypeRestriction( 
        xsd:nonNegativeInteger 
        xsd:minInclusive "1"^^xsd:nonNegativeInteger 
        xsd:maxInclusive "23"^^xsd:nonNegativeInteger ) )     
DataPropertyRange( :position :NumberPlayersPerWorldCupTeam ) 
\end{ex}

constraints (OWL 2 DL) [turtle syntax]:

\begin{ex}
:NumberPlayersPerWorldCupTeam
    a rdfs:Datatype ;
    owl:equivalentClass [
        a rdfs:Datatype ;
        owl:onDatatype xsd:nonNegativeInteger ;
        owl:withRestrictions ( 
            [ xsd:minInclusive "1"^^xsd:nonNegativeInteger ]
            [ xsd:maxInclusive "23"^^xsd:nonNegativeInteger ] ) ] .
\end{ex}

The data range 'NumberPlayersPerWorldCupTeam' contains the non negative integers 1 to 23, as each world cup team can only have 23 football players at most.

valid data (OWL 2 DL):

\begin{ex}
:MarioGoetze
    :position "19"^^:NumberPlayersPerWorldCupTeam .
\end{ex}

invalida data (OWL 2 DL):

\begin{ex}
:MarioGoetze
    :position "99"^^:NumberPlayersPerWorldCupTeam .
\end{ex}

\subsection{Language of RDF Literals}

requirements:

\begin{itemize}
	\item R-47-LANGUAGE-TAG-MATCHING
  \item R-48-MISSING-LANGUAGE-TAGS
	\item R-49-RDF-LITERALS-HAVING-AT-MOST-ONE-LANGUAGE-TAG
\end{itemize}

\subsubsection{R-49-RDF-LITERALS-HAVING-AT-MOST-ONE-LANGUAGE-TAG} 

constraints (DQTP):

ONELANGPattern \cite{Kontokostas2014} 

A literal value should contain at most
1 literal for a language. P1 is the property containing the
literal and V1 is the language we want to check.

\begin{ex}
SELECT DISTINCT ?s WHERE { ?s %%P1%% ?c
    BIND ( lang(?c) AS ?l )
    FILTER (isLiteral (?c) && lang(?c) = %%V1%%)}
GROUP BY ?s HAVING COUNT (?l) > 1
\end{ex}

test binding (DQTP):

a single English (“en”) foaf:name

\begin{ex}
P1 => foaf:name
V1 => en
\end{ex}

valid data (DQTP):

\begin{ex}
:LeiaSkywalker
    foaf:name 'Leia Skywalker'@en .
\end{ex}

invalid data (DQTP):

\begin{ex}
:LeiaSkywalker
    foaf:name 'Leia Skywalker'@en ;
    foaf:name 'Leia'@en .
\end{ex}

\subsection{Property Occurrences}

requirements:

\begin{itemize}
	\item R-52-NEGATIVE-OBJECT-PROPERTY-CONSTRAINTS
	\item R-53-NEGATIVE-DATA-PROPERTY-CONSTRAINTS
	\item R-67-CLASSIFY-PROPERTIES-ACCORDING-TO-OCCURRENCE
\end{itemize}

\subsubsection{R-52-NEGATIVE-OBJECT-PROPERTY-CONSTRAINTS}

\begin{itemize}
	\item instances of specific class must not have some object property
  \item OWL 2 DL: ObjectComplementOf ( ObjectSomeValuesFrom ( ObjectPropertyExpression owl:Thing ) )
	\item covering approaches: ShEx, SPIN
\end{itemize}

constraint (SPIN)

A matching triple has any predicate except those excluded by the '-' operator.

constraint (ShEx):

\begin{ex}
<FeelingForce> {
    :feelingForce 'true'^^xsd:boolean ,
    :attitute xsd:string }
<JediMentor> {
    :feelingForce 'true'^^xsd:boolean ,
    :attitute 'good'^^xsd:string ,
    :laserSwordColor xsd:string ,
    :numberLaserSwords xsd:nonNegativeInteger ,
    :mentorOf @<JediStudent> ,
    - :studentOf @<JediMentor> }
<JediStudent> {
    :feelingForce 'true'^^xsd:boolean ,
    :attitute 'good'^^xsd:string ,
    :laserSwordColor xsd:string ,
    :numberLaserSwords xsd:nonNegativeInteger ,
    - :mentorOf @<JediStudent> ,
    :studentOf @<JediMentor> }
\end{ex}

individuals matching 'FeelingForce' and 'JediMentor' data shapes:

\begin{ex}
:Obi-Wan 
    :feelingForce 'true'^^xsd:boolean ;
    :attitute 'good'^^xsd:string ;
    :laserSwordColor 'blue'^^xsd:string ,
    :numberLaserSwords '1'^^xsd:nonNegativeInteger ,
    :mentorOf :Anakin .
\end{ex}

individuals matching 'FeelingForce' and 'JediStudent' data shapes:

\begin{ex}
:Anakin 
    :feelingForce 'true'^^xsd:boolean ;
    :attitute 'good'^^xsd:string ;
    :laserSwordColor 'blue'^^xsd:string ,
    :numberLaserSwords '1'^^xsd:nonNegativeInteger ,
    :studentOf :Obi-Wan . 
\end{ex}

\subsubsection{R-53-NEGATIVE-DATA-PROPERTY-CONSTRAINTS}

\begin{itemize}
	\item analogous to R-52-NEGATIVE-OBJECT-PROPERTY-CONSTRAINTS
	\item covering approaches: ShEx, SPIN
\end{itemize}

\subsection{Property Groups}

requirements:

\begin{itemize}
	\item R-13-DISJOINT-GROUP-OF-PROPERTIES-CLASS-SPECIFIC
	\item R-66-PROPERTY-GROUPS
\end{itemize}

\subsection{RDF-Specific Validation}

requirements:

\begin{itemize}
	\item R-120-HANDLE-RDF-COLLECTIONS
\end{itemize}

examples:

\begin{itemize}
	\item size of collection
	\item first / last element of list must be a specific RDF literal
	\item compare elements of collection
	\item are collections identical?
	\item actions on RDF lists: \url{http://www.snee.com/bobdc.blog/2014/04/rdf-lists-and-sparql.html}
	\item 2. list element equals ''
	\item Does the list have more than 10 elements?
\end{itemize}

constraint (SPIN):

retrieves the 2. item from the list (2. student of Jedi mentor Jinn)

function call (SPIN):

\begin{ex}
BIND ( :getListItem( ?list, "1"xsd:nonNegativeInteger ) AS ?listItem ) .
\end{ex}

function (SPIN):

\begin{ex}
:getListItem
    a spin:Function ; rdfs:subClassOf spin:Functions ;
    spin:constraint [
        rdf:type spl:Argument ;
        spl:predicate sp:arg1 ;
        spl:valueType rdf:List ;
        rdfs:comment "list" ; ] ;
    spin:constraint [
        rdf:type spl:Argument ;
        spl:predicate sp:arg2 ;
        spl:valueType xsd:nonNegativeInteger ;
        rdfs:comment "item position (starting with 0)" ; ] ;
    spin:body [
        a sp:SELECT ;
        sp:text """
            SELECT ?item
            WHERE {
                ?arg1 :contents/rdf:rest{?arg2}/rdf:first ?item } """ ; ] ;
    spin:returnType rdfs:Resource .
\end{ex}

data:

\begin{ex}
:Jinn :students 
     ( :Xanatos :Kenobi ) . 
\end{ex}

result:

\begin{ex}
:Kenobi
\end{ex}

\subsection{RDF Shapes}

requirements:

\begin{itemize}
	\item R-125-RDF-SHAPE-CHECKING
\end{itemize}

\subsection{RDF Validation Results}

requirements:

\begin{itemize}
	\item R-150-RDF-REPRESENTATION-OF-VALIDATION-RESULTS
	\item R-151-USEFUL-MESSAGE-VALIDATION-RESULTS
	\item R-152-FIND-NOT-VALIDATED-TRIPLES
	\item R-153-RDF-REPRESENTATION-OF-CONSTRAINT-VIOLATIONS
	\item R-154-HANDLE-CONSTRAINT-VIOLATIONS
	\item R-155-GUIDANCE-HOW-TO-BECOME-VALID-DATA
	\item R-156-REFERENCES-TO-TRIPLES-CAUSING-THE-CONSTRAINT-VIOLATIONS
	\item R-157-REFERENCES-TO-VALIDATION-RULES-CAUSING-CONSTRAINT-VIOLATIONS
	\item R-158-SEVERITY-LEVELS-OF-CONSTRAINT-VIOLATIONS
	\item R-159-EXPLAIN-REASONS-OF-CONSTRAINT-VIOLATIONS
\end{itemize}

\subsection{Requirements Covered by OWL 2 DL But Not by OWL 2 QL}



\section{Related Work}



\section{Evaluation}



\subsection{Evaluation Using Practical Data Set}



\subsection{RDF Validation Requirements Covered by OWL 2 QL}

Table~\ref{tab:RequirementsCoveredOWL2QL}

\begin{table}
\caption{RDF Validation Requirements Covered by OWL 2 QL}
\label{tab:RequirementsCoveredOWL2QL}
\centering
\begin{tabular}{ll}
\hr
Requirements Classification & Requirements \\
\hr
& R-1-UNIQUENESS-OF-URIS \\
& R-2-UNIQUE-INSTANCES \\
& R-3-EQUIVALENT-CLASSES (EquivalentClasses) \\
& R-4-EQUIVALENT-OBJECT-PROPERTIES (EquivalentObjectProperties) \\
& R-5-EQUIVALENT-DATA-PROPERTIES (EquivalentDataProperties) \\
& R-7-DISJOINT-CLASSES (DisjointClasses) \\
& R-9-DISJOINT-OBJECT-PROPERTIES (DisjointObjectProperties) \\
& R-10-DISJOINT-DATA-PROPERTIES (DisjointDataProperties) \\
& R-14-DISJOINT-INDIVIDUALS (DifferentIndividuals) \\
& R-25-OBJECT-PROPERTY-DOMAIN (ObjectPropertyDomain) \\
& R-26-DATA-PROPERTY-DOMAIN (DataPropertyDomain) \\
& R-27-CLASS-SPECIFIC-VALIDATION \\
& R-28-OBJECT-PROPERTY-RANGE (ObjectPropertyRange) \\
& R-35-DATA-PROPERTY-RANGE (DataPropertyRange) \\
& R-54-SUB-OBJECT-PROPERTIES (SubObjectPropertyOf) \\
& R-56-INVERSE-OBJECT-PROPERTIES (ObjectInverseOf) \\
& R-59-REFLEXIVE-OBJECT-PROPERTIES (ReflexiveObjectProperty) \\
& R-60-IRREFLEXIVE-OBJECT-PROPERTIES (IrreflexiveObjectProperty) \\
& R-61-SYMMETRIC-OBJECT-PROPERTIES (SymmetricObjectProperty) \\
& R-62-ASYMMETRIC-OBJECT-PROPERTIES (AsymmetricObjectProperty) \\
& R-64-SUB-DATA-PROPERTIES (SubDataPropertyOf) \\
& R-93-DIFFERENCE-BETWEEN-CONSTRAINTS-ON-OBJECT-AND-DATA-PROPERTIES \\
& R-94-POSITIVE-OBJECT-PROPERTY-ASSERTIONS (ObjectPropertyAssertion) \\
& R-95-POSITIVE-DATA-PROPERTY-ASSERTIONS (DataPropertyAssertion) \\
& R-99-STABLE-IDENTIFICATION-OF-CONSTRAINTS \\
& R-100-SUBSUMPTION (SubClassOf) \\
& R-101-DECLARATIVE-CONSTRAINT-LANGUAGE \\
& R-102-INTUITIVE-CONSTRAINT-LANGUAGE \\
& R-103-HIGH-LEVEL-CONSTRAINT-LANGUAGE \\
& R-104-CONSTRAINT-LANGUAGE-HAVING-IMPLEMENTATION-LANGUAGE \\
& R-105-CONSTRAINT-LANGUAGE-TRANSLATABLE-TO-IMPLEMENTATION-LANGUAGE \\
& R-107-TRANSFORMATIONS-BETWEEN-CONSTRAINT-LANGUAGE-AND-UML \\
& R-108-TRANSFORMATIONS-BETWEEN-CONSTRAINT-LANGUAGE-AND-XML-SCHEMA \\
& R-109-TRANSFORMATIONS-BETWEEN-CONSTRAINT-LANGUAGE-AND-OCL \\
& R-110-TRANSFORMATIONS-BETWEEN-CONSTRAINT-LANGUAGE-AND-SPARQL \\
\hr
\end{tabular}
\end{table}

\begin{table}
\caption{RDF Validation Requirements Covered by OWL 2 QL}
\label{tab:RequirementsCoveredOWL2QL}
\centering
\begin{tabular}{ll}
\hr
Requirements Classification & Requirements \\
\hr
& R-111-BASIC-USE-CASES-COVERED-BY-CONSTRAINT-LANGUAGE \\
& R-113-INTERACTION-OF-VALIDATION-WITH-REASONING \\
& R-115-CLOSED-WORLD-ASSUMPTION-CWA \\
& R-116-UNIQUE-NAME-ASSUMPTION-UNA \\
& R-117-CONTEXT-SENSITIVE-CONSTRAINTS \\
& R-118-NAMESPACE-SENSITIVE-CONSTRAINTS \\
& R-122-TRADE-OFF-BETWEEN-DIMENSIONS-EXPRESSIVITY-COMPLEXITY-PREDICTABILITY \\
& R-124-DESCRIBE-DATA \\
& R-126-CUSTOMIZABLE-VALIDATION-PROCESS \\
& R-128-HUMAN-UNDERSTANDABLE-CONCRETE-SYNTAXES-FORMULATING-CONSTRAINTS \\
& R-129-MACHINE-UNDERSTANDABLE-CONCRETE-SYNTAXES-FORMULATION-CONSTRAINTS \\
& R-130-CONCISE-CONCRETE-SYNTAXES-FORMULATING-CONSTRAINTS \\
& R-131-OWL-AS-CONCRETE-SYNTAX-FORMULATING-CONSTRAINTS \\
& R-132-MULTIPLE-CONCRETE-SYNTAXES-FORMULATING-CONSTRAINTS \\
& R-133-MULTIPLE-CONCRETE-SYNTAXES-FORMULATING-DATA \\
& R-136-MODULARITY-OF-CONSTRAINT-DEFINITIONS \\
& R-137-LEVERAGE-ON-EXISTING-TECHNOLOGIES \\
& R-138-CONSTRAINT-LANGUAGE-COMPATIBLE-WITH-SPARQL \\
& R-139-CONSTRAINT-LANGUAGE-DRIVES-USER-INTERFACE-FORM-GENERATION-AND-PRESENTATION \\
& R-140-SEPARATE-ONTOLOGIES-FROM-VALIDATION-SCHEMAS \\
& R-143-CONDITIONAL-TYPED-VALIDATION \\
& R-147-DISTRIBUTION-OF-CONSTRAINT-SCHEMAS \\
& R-148-DISTRIBUTED-VALIDATION-IN-COLLABORATIVE-ENVIRONMENTS \\
& R-149-MANAGEMENT-OF-CONSTRAINT-SCHEMA-EVOLUTION \\
& R-160-OPEN-SOURCE-CONSTRAINT-VALIDATION \\
& R-161-ACCEPTABLE-PERFORMANCE-OF-VALIDATION-ALGORITHM \\
& R-162-SPECIFICATION-PUBLICLY-AVAILABLE \\
& R-163-IMPLEMENTATION-EXISTS \\
& R-164-IMPLEMENTATION-PUBLICLY-AVAILABLE \\
& R-165-EXECUTABLE-DEMOS-EXAMPLES-USE-CASES \\
& R-168-PERFORM-BIG-DATASETS \\
& R-169-RDF-REPRESENTATION-OF-CONSTRAINT-LANGUAGE \\
& R-173-SEPARATE-CONSTRAINTS-FROM-VOCABULARIES-AND-ONTOLOGIES \\
& R-174-REUSE-CONSTRAINTS \\
& R-175-DISCOVER-CONSTRAINTS \\
& R-177-DEFINE-SEMANTICS-FOR-CONSTRAINTS \\
& R-178-ASSOCIATE-CONSTRAINTS-WITH-VOCABULARIES \\
& R-182-USE-KNOWN-CONCRETE-SYNTAX \\
& R-184-COMPACT-CONCRETE-SYNTAX \\
& R-187-DEFINE-SEMANTICS-OF-CONSTRAINTS-IN-TERMS-OF-SPARQL \\
& R-190-SPECIFY-EXPECTED-BEHAVIOR-UNDER-ALL-POSSIBLE-ENTAILMENT-REGIMES \\
& R-192-DEFINE-ANNOTATIONS-FOR-CONSTRAINTS \\
& R-195-CONSTRAINT-LANGUAGE-EASY-TO-CONSUME-BY-TOOLS \\
& R-197-ATTACH-CONSTRAINTS-TO-CLASSES \\
& R-198-RDF-VALIDATION-AFTER-INFERENCING \\
& R-199-RDF-VALIDATION-MUST-COMPILE-DOWN-TO-SPARQL \\
\hr
\end{tabular}
\end{table}

\subsection{RDF Validation Requirements Not Covered by OWL 2 QL}

Table~\ref{tab:RequirementsNotCoveredOWL2QL-1}
Table~\ref{tab:RequirementsNotCoveredOWL2QL-2}
Table~\ref{tab:RequirementsNotCoveredOWL2QL-3}

%\medskip% adds some space before the table
%\begin{adjustwidth}{-1in}{-1in}% adjust the L and R margins by 1 inch
  %\begin{tabular}{ll}
    %\toprule
    %Requirements & Covering Constraint Languages \\
    %\midrule
%R-6-EQUIVALENT-INDIVIDUALS (SameIndividual) & OWL 2 DL \\
%R-8-DISJOINT-UNION-OF-CLASS-EXPRESSIONS (DisjointUnion) & OWL 2 DL \\
%R-11-DISJOINT-DATA-PROPERTIES-CLASS-SPECIFIC & ShEx \\
%R-12-DISJOINT-OBJECT-PROPERTIES-CLASS-SPECIFIC & ShEx \\
%R-13-DISJOINT-GROUP-OF-PROPERTIES-CLASS-SPECIFIC & ShEx \\
%R-15-CONJUNCTION-OF-CLASS-EXPRESSIONS (ObjectIntersectionOf) & OWL 2 DL \\
%R-16-CONJUNCTION-OF-DATA-RANGES (DataIntersectionOf) & OWL 2 DL \\
%R-17-DISJUNCTION-OF-CLASS-EXPRESSIONS (ObjectUnionOf) & OWL 2 DL \\
%R-18-DISJUNCTION-OF-DATA-RANGES (DataUnionOf) & OWL 2 DL \\
%R-19-NEGATION-OF-CLASS-EXPRESSIONS (ObjectComplementOf) & OWL 2 DL \\
%R-20-NEGATION-OF-DATA-RANGES (DataComplementOf) & OWL 2 DL \\
%R-21-IRI-PATTERN-MATCHING-ON-RDF-SUBJECTS & SPIN \\
%R-22-IRI-PATTERN-MATCHING-ON-RDF-OBJECTS & SPIN, ShEx \\
%R-23-IRI-PATTERN-MATCHING-ON-RDF-PROPERTIES & SPIN, ShEx \\
%R-24-PROVENANCE-CONSTRAINTS & \\
%R-29-CLASS-SPECIFIC-RANGE-OF-RDF-OBJECTS & SPIN, RS \\
%R-30-ALLOWED-VALUES-FOR-RDF-OBJECTS (ObjectOneOf) & OWL 2 DL \\
%R-31-DEFAULT-VALUES-OF-RDF-OBJECTS & SPIN, BF, RS \\
%R-32-MEMBERSHIP-OF-RDF-OBJECTS-IN-CONTROLLED-VOCABULARIES & DSP, SPIN \\
%R-33-NEGATIVE-OBJECT-CONSTRAINTS \\
%R-34-AVAILABLE-CLASS-DEFINITION & ShEx, SPIN \\
%R-36-CLASS-SPECIFIC-RANGE-OF-RDF-LITERALS & ShEx, BF, SPIN \\
%R-37-ALLOWED-VALUES-FOR-RDF-LITERALS (DataOneOf) & OWL 2 DL \\
%R-38-DEFAULT-VALUES-OF-RDF-LITERALS & SPIN, BF, RS \\
%R-39-MEMBERSHIP-OF-RDF-LITERALS-IN-CONTROLLED-VOCABULARIES & DSP, SPIN \\
%R-44-PATTERN-MATCHING-ON-RDF-LITERALS & ShEx, RS, SPIN \\
%R-41-STATISTICAL-COMPUTATIONS & SPIN \\
%R-42-COMPUTATIONS-BASED-ON-DATATYPE & \\
%R-43-COMPARISONS-BASED-ON-DATATYPE & DQTP, ShEx, SPIN \\
%R-45-RANGES-OF-RDF-LITERAL-VALUES & DQTP, SPIN \\
%R-46-CONSTRAINING-FACETS & Stardog, SPIN \\
%R-47-LANGUAGE-TAG-MATCHING & SPIN \\
%R-48-MISSING-LANGUAGE-TAGS & SPIN \\
%R-49-RDF-LITERALS-HAVING-AT-MOST-ONE-LANGUAGE-TAG & DQTP, SPIN \\
%R-50-WHITESPACE-HANDLING-OF-RDF-LITERALS & SPIN \\
%R-51-HTML-HANDLING-OF-RDF-LITERALS & SPIN \\
%R-52-NEGATIVE-OBJECT-PROPERTY-CONSTRAINTS & ShEx, SPIN \\
%R-53-NEGATIVE-DATA-PROPERTY-CONSTRAINTS & ShEx, SPIN \\
%R-55-OBJECT-PROPERTY-PATHS (SubObjectPropertyOf) & OWL 2 DL \\
%R-57-FUNCTIONAL-OBJECT-PROPERTIES (FunctionalObjectProperty) & OWL 2 DL \\
%R-58-INVERSE-FUNCTIONAL-OBJECT-PROPERTIES (InverseFunctionalObjectProperty) & OWL 2 DL \\
%R-63-TRANSITIVE-OBJECT-PROPERTIES (TransitiveObjectProperty) & OWL 2 DL \\
%R-65-FUNCTIONAL-DATA-PROPERTIES (FunctionalDataProperty) & OWL 2 DL \\
%R-66-PROPERTY-GROUPS & ShEx, SPIN \\
%R-67-CLASSIFY-PROPERTIES-ACCORDING-TO-OCCURRENCE & \\
%R-68-REQUIRED-PROPERTIES (ObjectMinCardinality, DataMinCardinality) & OWL 2 DL \\
%R-69-OPTIONAL-PROPERTIES  (ObjectMinCardinality, DataMinCardinality) & OWL 2 DL \\
%R-70-REPEATABLE-PROPERTIES (ObjectMinCardinality, DataMinCardinality) & OWL 2 DL \\
%R-71-CONDITIONAL-PROPERTIES \\
%R-72-RECOMMENDED-PROPERTIES \\
%R-74-EXACT-QUALIFIED-CARDINALITY-RESTRICTIONS-ON-OBJECT-PROPERTIES (ObjectExactCardinality) & OWL 2 DL \\
%R-75-MINIMUM-QUALIFIED-CARDINALITY-RESTRICTIONS-ON-OBJECT-PROPERTIES (ObjectMinCardinality) & OWL 2 DL \\
%R-141-NEGATIVE-PATTERN-MATCHING-ON-RDF-LITERALS & DQTP, SPIN \\
%R-142-NEGATIVE-RANGES-OF-RDF-LITERAL-VALUES & DQTP, SPIN \\
    %\bottomrule
  %\end{tabular}
%\end{adjustwidth}
%\medskip% adds some space after the table

\begin{table}
\caption{RDF Validation Requirements Not Covered by OWL 2 QL}
\label{tab:RequirementsNotCoveredOWL2QL-1}
\centering
\begin{tabular}{ll}
\hr
Requirements & Covering Constraint Languages \\
\hr
R-6-EQUIVALENT-INDIVIDUALS (SameIndividual) & OWL 2 DL \\
R-8-DISJOINT-UNION-OF-CLASS-EXPRESSIONS (DisjointUnion) & OWL 2 DL \\
R-11-DISJOINT-DATA-PROPERTIES-CLASS-SPECIFIC & ShEx \\
R-12-DISJOINT-OBJECT-PROPERTIES-CLASS-SPECIFIC & ShEx \\
R-13-DISJOINT-GROUP-OF-PROPERTIES-CLASS-SPECIFIC & ShEx \\
R-15-CONJUNCTION-OF-CLASS-EXPRESSIONS (ObjectIntersectionOf) & OWL 2 DL \\
R-16-CONJUNCTION-OF-DATA-RANGES (DataIntersectionOf) & OWL 2 DL \\
R-17-DISJUNCTION-OF-CLASS-EXPRESSIONS (DisjointUnionOf) & OWL 2 DL \\
R-18-DISJUNCTION-OF-DATA-RANGES (DataUnionOf) & OWL 2 DL \\
R-19-NEGATION-OF-CLASS-EXPRESSIONS (ObjectComplementOf) & OWL 2 DL \\
R-20-NEGATION-OF-DATA-RANGES (DataComplementOf) & OWL 2 DL \\
R-21-IRI-PATTERN-MATCHING-ON-RDF-SUBJECTS & SPIN \\
R-22-IRI-PATTERN-MATCHING-ON-RDF-OBJECTS & SPIN, ShEx \\
R-23-IRI-PATTERN-MATCHING-ON-RDF-PROPERTIES & SPIN, ShEx \\
R-24-PROVENANCE-CONSTRAINTS & \\
R-29-CLASS-SPECIFIC-RANGE-OF-RDF-OBJECTS & SPIN, RS \\
R-30-ALLOWED-VALUES-FOR-RDF-OBJECTS (ObjectOneOf) & OWL 2 DL \\
R-31-DEFAULT-VALUES-OF-RDF-OBJECTS & SPIN, BF, RS \\
R-32-MEMBERSHIP-OF-RDF-OBJECTS-IN-CONTROLLED-VOCABULARIES & DSP, SPIN \\
R-33-NEGATIVE-OBJECT-CONSTRAINTS \\
R-34-AVAILABLE-CLASS-DEFINITION & ShEx, SPIN \\
R-36-CLASS-SPECIFIC-RANGE-OF-RDF-LITERALS & ShEx, BF, SPIN \\
R-37-ALLOWED-VALUES-FOR-RDF-LITERALS (DataOneOf) & OWL 2 DL \\
R-38-DEFAULT-VALUES-OF-RDF-LITERALS & SPIN, BF, RS \\
R-39-MEMBERSHIP-OF-RDF-LITERALS-IN-CONTROLLED-VOCABULARIES & DSP, SPIN \\
R-44-PATTERN-MATCHING-ON-RDF-LITERALS & DQTP, OWL 2 DL, ShEx, RS, SPARQL, SPIN \\
R-41-STATISTICAL-COMPUTATIONS & SPIN \\
R-42-COMPUTATIONS-BASED-ON-DATATYPE & \\
R-43-COMPARISONS-BASED-ON-DATATYPE & DQTP, ShEx, SPIN \\
R-45-RANGES-OF-RDF-LITERAL-VALUES & DQTP, SPIN \\
R-46-CONSTRAINING-FACETS & Stardog, SPIN \\
R-47-LANGUAGE-TAG-MATCHING & SPIN \\
R-48-MISSING-LANGUAGE-TAGS & SPIN \\
R-49-RDF-LITERALS-HAVING-AT-MOST-ONE-LANGUAGE-TAG & DQTP, SPIN \\
R-50-WHITESPACE-HANDLING-OF-RDF-LITERALS & SPIN \\
R-51-HTML-HANDLING-OF-RDF-LITERALS & SPIN \\
R-52-NEGATIVE-OBJECT-PROPERTY-CONSTRAINTS & ShEx, SPIN \\
R-53-NEGATIVE-DATA-PROPERTY-CONSTRAINTS & ShEx, SPIN \\
R-55-OBJECT-PROPERTY-PATHS (SubObjectPropertyOf) & OWL 2 DL \\
R-57-FUNCTIONAL-OBJECT-PROPERTIES (FunctionalObjectProperty) & OWL 2 DL \\
R-58-INVERSE-FUNCTIONAL-OBJECT-PROPERTIES (InverseFunctionalObjectProperty) & OWL 2 DL \\
R-63-TRANSITIVE-OBJECT-PROPERTIES (TransitiveObjectProperty) & OWL 2 DL \\
R-65-FUNCTIONAL-DATA-PROPERTIES (FunctionalDataProperty) & OWL 2 DL \\
R-66-PROPERTY-GROUPS & ShEx, SPIN \\
R-67-CLASSIFY-PROPERTIES-ACCORDING-TO-OCCURRENCE & \\
R-68-REQUIRED-PROPERTIES (ObjectMinCardinality, DataMinCardinality) & OWL 2 DL \\
R-69-OPTIONAL-PROPERTIES  (ObjectMinCardinality, DataMinCardinality) & OWL 2 DL \\
R-70-REPEATABLE-PROPERTIES (ObjectMinCardinality, DataMinCardinality) & OWL 2 DL \\
R-71-CONDITIONAL-PROPERTIES \\
R-72-RECOMMENDED-PROPERTIES \\
\hr
\end{tabular}
\end{table}

\begin{table}
\caption{RDF Validation Requirements Not Covered by OWL 2 QL}
\label{tab:RequirementsNotCoveredOWL2QL-2}
\centering
\begin{tabular}{ll}
\hr
Requirements & Covering Constraint Languages \\
\hr
R-74-EXACT-QUALIFIED-CARDINALITY-RESTRICTIONS-ON-OBJECT-PROPERTIES (ObjectExactCardinality) & OWL 2 DL \\
R-75-MINIMUM-QUALIFIED-CARDINALITY-RESTRICTIONS-ON-OBJECT-PROPERTIES (ObjectMinCardinality) & OWL 2 DL \\
R-76-MAXIMUM-QUALIFIED-CARDINALITY-RESTRICTIONS-ON-OBJECT-PROPERTIES (ObjectMaxCardinality) & OWL 2 DL \\
R-77-EXACT-QUALIFIED-CARDINALITY-RESTRICTIONS-ON-DATA-PROPERTIES (DataExactCardinality) & OWL 2 DL \\
R-78-MINIMUM-QUALIFIED-CARDINALITY-RESTRICTIONS-ON-DATA-PROPERTIES (DataMinCardinality) & OWL 2 DL \\
R-79-MAXIMUM-QUALIFIED-CARDINALITY-RESTRICTIONS-ON-DATA-PROPERTIES (DataMaxCardinality) & OWL 2 DL \\
R-80-EXACT-UNQUALIFIED-CARDINALITY-RESTRICTIONS-ON-OBJECT-PROPERTIES (ObjectExactCardinality) & OWL 2 DL \\
R-81-MINIMUM-UNQUALIFIED-CARDINALITY-RESTRICTIONS-ON-OBJECT-PROPERTIES (ObjectMinCardinality) & OWL 2 DL \\
R-82-MAXIMUM-UNQUALIFIED-CARDINALITY-RESTRICTIONS-ON-OBJECT-PROPERTIES (ObjectMaxCardinality) & OWL 2 DL \\
R-83-EXACT-UNQUALIFIED-CARDINALITY-RESTRICTIONS-ON-DATA-PROPERTIES (DataExactCardinality) & OWL 2 DL \\
R-84-MINIMUM-UNQUALIFIED-CARDINALITY-RESTRICTIONS-ON-DATA-PROPERTIES (DataMinCardinality) & OWL 2 DL \\
R-85-MAXIMUM-UNQUALIFIED-CARDINALITY-RESTRICTIONS-ON-DATA-PROPERTIES (DataMaxCardinality) & OWL 2 DL \\
\hr
\end{tabular}
\end{table}

\begin{table}
\caption{RDF Validation Requirements Not Covered by OWL 2 QL}
\label{tab:RequirementsNotCoveredOWL2QL-3}
\centering
\begin{tabular}{ll}
\hr
Requirements & Covering Constraint Languages \\
\hr
R-86-EXISTENTIAL-QUANTIFICATION-ON-OBJECT-PROPERTIES (ObjectSomeValuesFrom) & OWL 2 DL \\
R-87-UNIVERSAL-QUANTIFICATION-ON-OBJECT-PROPERTIES (ObjectAllValuesFrom) & OWL 2 DL \\
R-88-INDIVIDUAL-VALUE-RESTRICTION-ON-OBJECT-PROPERTIES (ObjectHasValue) & OWL 2 DL \\
R-89-SELF-RESTRICTION (ObjectHasSelf) & OWL 2 DL \\
R-90-EXISTENTIAL-QUANTIFICATION-ON-DATA-PROPERTIES (DataSomeValuesFrom) & OWL 2 DL \\
R-91-UNIVERSAL-QUANTIFICATION-ON-DATA-PROPERTIES (DataAllValuesFrom) & OWL 2 DL \\
R-92-LITERAL-VALUE-RESTRICTION (DataHasValue) & OWL 2 DL \\
R-96-NEGATIVE-OBJECT-PROPERTY-ASSERTIONS (NegativeObjectPropertyAssertion) & OWL 2 DL \\
R-97-NEGATIVE-DATA-PROPERTY-ASSERTIONS (NegativeDataPropertyAssertion) & OWL 2 DL \\
R-98-CHECK-VALIDITY-OF-URIS & \\
R-106-EXTENSIBLE-CONSTRAINT-LANGUAGE & \\
R-112-EXTENSIBLE-CONSTRAINTS & \\
R-114-PROVIDE-RDF-REST-SERVICES-FOR-RDF-VALIDATION & \\
R-119-VALIDATION-ON-NAMED-GRAPHS & \\
R-120-HANDLE-RDF-COLLECTIONS & ShEx, SPIN \\
R-121-SPECIFY-ORDER-OF-RDF-RESOURCES & \\
R-123-STATE & \\
R-125-RDF-SHAPE-CHECKING & ShEx \\
R-126-CUSTOMIZABLE-VALIDATION-PROCESS & \\
R-134-SPECIFY-USAGE-OF-TERMS & \\
R-135-CONSTRAINT-LEVELS & \\
R-141-NEGATIVE-PATTERN-MATCHING-ON-RDF-LITERALS & DQTP, OWL 2 DL, SPARQL, SPIN \\
R-142-NEGATIVE-RANGES-OF-RDF-LITERAL-VALUES & DQTP, SPIN \\
R-146-CONSTRAINT-VALIDATION-OF-RDF-INPUT-WITH-RESPECT-TO-EXISTING-RDF & \\
R-150-RDF-REPRESENTATION-OF-VALIDATION-RESULTS & SPIN, DQTP \\
R-151-USEFUL-MESSAGE-VALIDATION-RESULTS & SPIN, DQTP, ShEx, Pellet \\
R-152-FIND-NOT-VALIDATED-TRIPLES & ShEx \\
R-153-RDF-REPRESENTATION-OF-CONSTRAINT-VIOLATIONS & SPIN \\
R-154-HANDLE-CONSTRAINT-VIOLATIONS & ShEx, SPIN \\
R-155-GUIDANCE-HOW-TO-BECOME-VALID-DATA & ShEx, SPIN \\
R-156-REFERENCES-TO-TRIPLES-CAUSING-THE-CONSTRAINT-VIOLATIONS & SPIN \\
R-157-REFERENCES-TO-VALIDATION-RULES-CAUSING-CONSTRAINT-VIOLATIONS & ShEx, SPIN \\
R-158-SEVERITY-LEVELS-OF-CONSTRAINT-VIOLATIONS & SPIN \\
R-159-EXPLAIN-REASONS-OF-CONSTRAINT-VIOLATIONS & SPIN, ShEx, Stardog, Pellet \\
R-166-RDF-STREAMING-VALIDATION & \\
R-167-VALIDATE-RDF-IN-AN-HTML-DOCUMENT-CONSTAINING-RDFA-MARKUP & \\
R-170-VALIDATION-OF-SPARQL-ENDPOINTS & \\
R-171-VALIDATION-OF-URIS-BY-DEREFERENCING & \\
R-172-GENERATE-HUMAN-READABLE-DOCUMENTATION & \\
R-176-PROVIDE-HIGH-LEVEL-VOCABULARY-FOR-THE-MOST-COMMON-TYPES-OF-CONSTRAINTS & \\
R-179-ASSOCIATE-CONSTRAINTS-WITH-RDF-DOCUMENTS & \\
R-180-ASSOCIATE-CONSTRAINTS-WITH-RDF-DATASETS & \\
R-181-ASSOCIATE-CONSTRAINTS-WITH-RDF-REST-APIS & \\
R-183-CONSTRAINTS-ABOUT-CONSTRAINTS & \\
R-185-FEDERALIZED-RDF-VALIDATION & \\
R-186-EXTEND-EXPRESSIVITY-WITH-SPARQL & \\
R-188-EXPRESSIVITY-OF-CONSTRAINT-LANGUAGE-EQUIVALENT-TO-SPARQL & \\
R-189-ADD-ANNOTATIONS-TO-CONSTRAINT-VIOLATION-OBJECTS & \\
R-191-SHAPES-RELATED-TO-TYPES & \\
R-193-MULTIPLE-CONSTRAINT-VALIDATION-EXECUTION-LEVELS & \\
R-194-PROVIDE-STRING-FUNCTIONS-FOR-RDF-LITERALS & SPIN \\
\hr
\end{tabular}
\end{table}

\section{Conclusion and Future Work}

\section{Appendix}


\subsection{Allowed Usage of Constructs in Class Expressions in OWL 2 QL}

\textbf{Subclass Expressions}

\begin{ex}
subClassExpression :=
    Class |
    subObjectSomeValuesFrom | DataSomeValuesFrom
subObjectSomeValuesFrom := 'ObjectSomeValuesFrom' '(' ObjectPropertyExpression owl:Thing ')'
\end{ex}

\textbf{Superclass Expressions}

\begin{ex}
superClassExpression :=
    Class |
    superObjectIntersectionOf | superObjectComplementOf |
    superObjectSomeValuesFrom | DataSomeValuesFrom
\end{ex}
 
\subsection{Supported Constructs in OWL 2 QL}

\begin{itemize}
	\item subclass axioms (SubClassOf)
  \item class expression equivalence (EquivalentClasses)
  \item class expression disjointness (DisjointClasses)
  \item inverse object properties (InverseObjectProperties)
  \item property inclusion (SubObjectPropertyOf not involving property chains and SubDataPropertyOf)
  \item property equivalence (EquivalentObjectProperties and EquivalentDataProperties)
  \item property domain (ObjectPropertyDomain and DataPropertyDomain)
  \item property range (ObjectPropertyRange and DataPropertyRange)
  \item disjoint properties (DisjointObjectProperties and DisjointDataProperties)
  \item symmetric properties (SymmetricObjectProperty)
  \item reflexive properties (ReflexiveObjectProperty)
  \item irreflexive properties (IrreflexiveObjectProperty)
  \item asymmetric properties (AsymmetricObjectProperty)
  \item assertions: DifferentIndividuals, ClassAssertion, ObjectPropertyAssertion, and DataPropertyAssertion
\end{itemize}

\subsection{Not Supported Constructs in OWL 2 QL}

\begin{itemize}
	\item existential quantification to a class expression or a data range (ObjectSomeValuesFrom and DataSomeValuesFrom) in the subclass position
  \item self-restriction (ObjectHasSelf)
  \item existential quantification to an individual or a literal (ObjectHasValue, DataHasValue)
  \item enumeration of individuals and literals (ObjectOneOf, DataOneOf)
  \item universal quantification to a class expression or a data range (ObjectAllValuesFrom, DataAllValuesFrom)
  \item cardinality restrictions (ObjectMaxCardinality, ObjectMinCardinality, ObjectExactCardinality, DataMaxCardinality, DataMinCardinality, DataExactCardinality)
  \item disjunction (ObjectUnionOf, DisjointUnion, and DataUnionOf)
  \item property inclusions (SubObjectPropertyOf) involving property chains
  \item functional and inverse-functional properties (FunctionalObjectProperty, InverseFunctionalObjectProperty, and FunctionalDataProperty)
  \item transitive properties (TransitiveObjectProperty)
  \item keys (HasKey)
  \item individual equality assertions and negative property assertions (SameIndividual, NegativeObjectPropertyAssertion, NegativeDataPropertyAssertion)
\end{itemize}



\bibliography{literatur}{}
\bibliographystyle{plain}
\setcounter{tocdepth}{1}
%\listoftodos
\end{document}
